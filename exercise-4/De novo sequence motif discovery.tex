\documentclass[a4paper,10pt,titlepage]{article}

\usepackage[utf8]{inputenc}
\usepackage[T1]{fontenc}
\usepackage[danish]{babel}
\usepackage{amssymb}
%\usepackage{mathtools}
\usepackage{bchart}
\usepackage{color}
\usepackage{xcolor}
\usepackage{listings}
\usepackage{float}
\usepackage{hyperref}
\hypersetup{%
    pdfborder = {0 0 0}
}

\parindent0em

\lstset{%
frame=single,
numbers=left,
numberstyle=\footnotesize,
tabsize=2,
keepspaces=true,
columns=fullflexible,
basicstyle=\ttfamily\scriptsize,
inputencoding=utf8,
extendedchars=true,
}


\begin{document}
\section*{Exercise sheet 4}

Group: Anders Busch, Dan Sebastian Thrane, Frederik Hertzum, Henrik Schulz, Lars Thomasen and Peter Gottlieb

%%%%%%%%%%%%%%%%%%%%%%%%%%%%%%%%%%%%%%%%%%%%%%%%%%%%%%%%%%%%%%
% TASK 1
%%%%%%%%%%%%%%%%%%%%%%%%%%%%%%%%%%%%%%%%%%%%%%%%%%%%%%%%%%%%%%
\subsection*{Explain/define: “Sequence motif” and “gene/transcription factor knock-out”}

\begin{itemize}
\item
In genetics, a \textbf{sequence motif} is a nucleotide or amino-acid sequence pattern that is widespread and has, or is conjectured to have, a biological significance. For proteins, a sequence motif is distinguished from a structural motif, a motif formed by the three-dimensional arrangement of amino acids, which may not be adjacent.
\item
Transcription factor knock-out, is the process of making a transcription factor inactive such that the regulatory effects of the knock-outed transcription factor can be observed. This is typically used to learn the functionallity of a lesser-known transcription factor.
\end{itemize}

%%%%%%%%%%%%%%%%%%%%%%%%%%%%%%%%%%%%%%%%%%%%%%%%%%%%%%%%%%%%%%
% TASK 2
%%%%%%%%%%%%%%%%%%%%%%%%%%%%%%%%%%%%%%%%%%%%%%%%%%%%%%%%%%%%%%
\subsection*{Imagine we performed at transcription factor (TF) knock-out study and identified 25 differentially expressed genes. Now, we aim to identify a binding motif for the knocked-out TF in the upstream sequences of these 25 genes de novo. We further assume the TF to dock a 19-bp sequence within each of the upstream sequences. }

\begin{itemize}
\item
\textbf{todo}
\end{itemize}

\subsubsection{Download the upstream sequences in FASTA format from URL1 (see below). Write a JAVA program SEQMOTIF that implements an Expectation Maximization algorithm on DNA sequences. What are the 25 most likely 19-bp binding sequences of the TF?}

For simplification: You may use the nucleotide content of all 25 upstream sequences as background distribution, i.e. you don’t need to update the background model; assume it’s static.

\begin{itemize}
\item
\textbf{todo}
\end{itemize}

\subsubsection{2b. Afterwards, use the sequence logo painter from our exercise sheet or the publicly available WebLogo painter (URL2, see below) to paint the sequence logo of the binding motif.}

\begin{itemize}
\item
\textbf{todo}
\end{itemize}

\subsubsection{2c. Give the consensus sequence.}

Hint: The consensus sequence of the most likely motif should start with TTAGG and end with CCTAA.

\begin{itemize}
\item
\textbf{todo}
\end{itemize}

\end{document}
