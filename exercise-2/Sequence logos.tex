\documentclass[a4paper,10pt,titlepage]{article}

\usepackage[utf8]{inputenc}
\usepackage[T1]{fontenc}
\usepackage[danish]{babel}
\usepackage{amssymb}
%\usepackage{mathtools}
\usepackage{bchart}
\usepackage{color}
\usepackage{xcolor}
\usepackage{listings}
\usepackage{float}
\usepackage{hyperref}
\hypersetup{%
    pdfborder = {0 0 0}
}

\parindent0em

\lstset{%
frame=single,
numbers=left,
numberstyle=\footnotesize,
tabsize=2,
keepspaces=true,
columns=fullflexible,
basicstyle=\ttfamily\scriptsize,
inputencoding=utf8,
extendedchars=true,
}


\begin{document}
Group: Anders Busch, Dan Sebastian Thrane, Frederik Hertzum, Henrik Schulz, Lars Thomasen and Peter Gottlieb
\begin{enumerate}

%%%%%%%%%%%%%%%%%%%%%%%%%%%%%%%%%%%%%%%%%%%%%%%%%%%%%%%%%%%%%%
% TASK 1
%%%%%%%%%%%%%%%%%%%%%%%%%%%%%%%%%%%%%%%%%%%%%%%%%%%%%%%%%%%%%%
\item
\textit{Explain/define: “Operon” and “Transcription Unit”?}\\

\begin{itemize}
\item
\textbf{An operon} is a functioning unit of genomic DNA containing a cluster of genes under the control of a single region of DNA that initiates transcription of a particular gene. 
\item
\textbf{Transcription} is a process that occurs in all living cells. During transcription, strands of RNA are created based on the DNA found within the cells. //
When a strand of messenger RNA (mRNA) is created, it is then used to produce proteins during translation. A whole strand of DNA is not usually transcribed into mRNA, but instead specific sections of the DNA are, which are called transcription units.
\end{itemize}

%%%%%%%%%%%%%%%%%%%%%%%%%%%%%%%%%%%%%%%%%%%%%%%%%%%%%%%%%%%%%%
% TASK 2
%%%%%%%%%%%%%%%%%%%%%%%%%%%%%%%%%%%%%%%%%%%%%%%%%%%%%%%%%%%%%%
\item
\textit{Write a Java program SELO that reads a file with the following eight sequences. SELO shall compute and output a table with the
characteristic numbers for a “standard” sequence logo (no HMM logo).}\\

\begin{itemize}
\item
See seperate java folder.
\end{itemize}

%%%%%%%%%%%%%%%%%%%%%%%%%%%%%%%%%%%%%%%%%%%%%%%%%%%%%%%%%%%%%%
% TASK 3
%%%%%%%%%%%%%%%%%%%%%%%%%%%%%%%%%%%%%%%%%%%%%%%%%%%%%%%%%%%%%%
\item
\textit{What is the main idea behind the unsupervised operon prediction method introduced in the class, i.e. why does it “work” without prior knowledge about concrete previously identified operons (sample/training data)?}\\

\begin{itemize}
\item
TODO
\end{itemize}

%%%%%%%%%%%%%%%%%%%%%%%%%%%%%%%%%%%%%%%%%%%%%%%%%%%%%%%%%%%%%%
% TASK 4
%%%%%%%%%%%%%%%%%%%%%%%%%%%%%%%%%%%%%%%%%%%%%%%%%%%%%%%%%%%%%%
\item
\textit{Why are the features for adjacent genes computed separately for closely and distantly related organisms?}\\

\begin{itemize}
\item
TODO
\end{itemize}


\end{enumerate}
\end{document}