\documentclass[a4paper,10pt,titlepage]{article}

\usepackage[utf8]{inputenc}
\usepackage[T1]{fontenc}
\usepackage[danish]{babel}
\usepackage{amssymb}
%\usepackage{mathtools}
\usepackage{bchart}
\usepackage{color}
\usepackage{xcolor}
\usepackage{listings}
\usepackage{float}
\usepackage{hyperref}
\hypersetup{%
    pdfborder = {0 0 0}
}

\parindent0em

\lstset{%
frame=single,
numbers=left,
numberstyle=\footnotesize,
tabsize=2,
keepspaces=true,
columns=fullflexible,
basicstyle=\ttfamily\scriptsize,
inputencoding=utf8,
extendedchars=true,
}


\begin{document}
\section*{Exercise sheet 2}

Group: Anders Busch, Dan Sebastian Thrane, Frederik Hertzum, Henrik Schulz, Lars Thomasen and Peter Gottlieb

%%%%%%%%%%%%%%%%%%%%%%%%%%%%%%%%%%%%%%%%%%%%%%%%%%%%%%%%%%%%%%
% TASK 1
%%%%%%%%%%%%%%%%%%%%%%%%%%%%%%%%%%%%%%%%%%%%%%%%%%%%%%%%%%%%%%
\subsection*{Explain/define: “Transcription Factor Binding Site” and “Position Weight Matrix”.}\\

\begin{itemize}
\item
\textbf{Transcription Factor (Binding Site)} is a protein that binds to specific DNA sequences, thereby controlling the rate of transcription of genetic information from DNA to messenger RNA.
\item
\textbf{A position weight matrix} (PWM), also known as a position-specific weight matrix (PSWM) or position-specific scoring matrix (PSSM), is a commonly used representation of motifs (sequence patterns) in biological sequences.
\end{itemize}

%%%%%%%%%%%%%%%%%%%%%%%%%%%%%%%%%%%%%%%%%%%%%%%%%%%%%%%%%%%%%%
% TASK 2
%%%%%%%%%%%%%%%%%%%%%%%%%%%%%%%%%%%%%%%%%%%%%%%%%%%%%%%%%%%%%%
\subsection*{Given are the following 18 binding sequences for a transcription
factor.}\\

\begin{verbatim}
cctacgcccc   
cctccttgcc
cctccacccc   
cctcctcccc
catcctcccg   
catcctcccg
cctccttgcc   
cctacgcccc
cctcctcccc   
cctccacccc
actcatcatc   
cctcctcccc
tatccgcccc   
tctcatcctg
actcatccct   
gctcaccctt
cctcatcctg   
actcctccct
\end{verbatim}

\begin{itemize}
\item
\textbf{Compute (1) a position count matrix} \\
\begin{equation}
  \begin{pmatrix}
	  & 0  & 1  & 2  & 3  & 4  & 5  & 6  & 7  & 8  & 9 \\
	\hline
	A & 3  & 3  & 0  & 2  & 5  & 2  & 0  & 1  & 0  & 0 \\
	C & 12 & 15 & 0  & 16 & 13 & 1  & 16 & 15 & 14 & 11\\
	G & 1  & 0  & 0  & 0  & 0  & 3  & 0  & 2  & 0  & 4 \\
	T & 2  & 0  & 18 & 0  & 0  & 12 & 2  & 0  & 4  & 3 \\
  \end{pmatrix}
\end{equation}

\item
\textbf{(2) a position weight matrix} \\
\begin{equation}
  \begin{pmatrix}
	  & 0  & 1  & 2  & 3  & 4  & 5  & 6  & 7  & 8  & 9 \\
	\hline
	A & 3/18  & 3/18  & 0/18  & 2/18  & 5/18  & 2/18  & 0/18  & 1/18  & 0/18  & 0/18 \\
	C & 12/18 & 15/18 & 0/18  & 16/18 & 13/18 & 1/18  & 16/18 & 15/18 & 14/18 & 11/18\\
	G & 1/18  & 0/18  & 0/18  & 0/18  & 0/18  & 3/18  & 0/18  & 2/18  & 0/18  & 4/18 \\
	T & 2/18  & 0/18  & 18/18 & 0/18  & 0/18  & 12/18 & 2/18  & 0/18  & 4/18  & 3/18 \\
  \end{pmatrix}
\end{equation}
Note that each entry should be multiplied by a pseudo number, to avoid having 0's. Having any 0's would cause the probability to be 0, regardless of the other entries.
We should assume the following background distribution of nucleotides.
\begin{verbatim}
A: 0.2 T: 0.2 C: 0.3 G: 0.3
\end{verbatim}
Therefore the PWM looks like:
\begin{equation}
  \begin{pmatrix}
	  & 0  & 1  & 2  & 3  & 4  & 5  & 6  & 7  & 8  & 9 \\
	\hline
	A & 3/18  & 3/18  & 0.2/18  & 2/18  & 5/18  & 2/18  & 0.2/18  & 1/18  & 0.2/18  & 0.2/18 \\
	C & 12/18 & 15/18 & 0.2/18  & 16/18 & 13/18 & 1/18  & 16/18 & 15/18 & 14/18 & 11/18\\
	G & 1/18  & 0.3/18  & 0.3/18  & 0.3/18  & 0.3/18  & 3/18  & 0.3/18  & 2/18  & 0.3/18  & 4/18 \\
	T & 2/18  & 0.3/18  & 18/18 & 0.3/18  & 0.3/18  & 12/18 & 2/18  & 0.3/18  & 4/18  & 3/18 \\
  \end{pmatrix}
\end{equation}

\end{itemize}

%%%%%%%%%%%%%%%%%%%%%%%%%%%%%%%%%%%%%%%%%%%%%%%%%%%%%%%%%%%%%%
% TASK 3
%%%%%%%%%%%%%%%%%%%%%%%%%%%%%%%%%%%%%%%%%%%%%%%%%%%%%%%%%%%%%%
\subsection*{-}\\

\begin{itemize}
\item
TODO
\end{itemize}

%%%%%%%%%%%%%%%%%%%%%%%%%%%%%%%%%%%%%%%%%%%%%%%%%%%%%%%%%%%%%%
% TASK 4
%%%%%%%%%%%%%%%%%%%%%%%%%%%%%%%%%%%%%%%%%%%%%%%%%%%%%%%%%%%%%%
\subsection*{-}\\

\begin{itemize}
\item
TODO
\end{itemize}


\end{enumerate}
\end{document}